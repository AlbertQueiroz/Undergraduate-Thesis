\chapter{INTRODUÇÃO}
\thispagestyle{empty}

Devido ao avanço do ensino a distância, cresceu a necessidade de criar ferramentas e recursos que fossem capazes de atrair a atenção dos alunos, principalmente os de idade menos elevada. Dado a evolução tecnológica e o contato cada vez mais frequente de crianças e jovens com aparelhos digitais, é inevitável anular a possibilidade de utilizar essas ferramentas como objetos de ensino. Visto isso, são inumeras as possibilidades para se aproveitar desses equipamentos.

Neste âmbito, foram realizadas diversas pesquisas relacionadas a eficácia da chamada "gamificação" no processo de aprendizagem, um exemplo de gamificação de questionários são os jogos de Quiz, que tem se tornado bastante populares e permitem abordar diferentes tipo de tema de forma lúdica e descontraída, como é mostrado por Carneiro (2014) em "Learning objects as enablers in distance education"\nocite{carneiro2014learning}, Vargas (2017) em "O processo de aprendizagem e avaliação através de
quiz"\nocite{vargas2017processo} e Bastos (2020) em "Quiz como ferramenta motivacional e avaliativa no ensino-aprendizagem de química."\nocite{bastos2020quiz}

Pensando nisso, alunos do IFCE, Campus Canindé, pesquisaram e desenvolveram um jogo quiz voltado para educação financeira de crianças na região rural, denominado Grana, 2018\nocite{pereira2018grana}. Após o desenvolvimento do projeto foram realizados estudos focados em analisar a sua eficácia no auxilio do ensino de gestão financeira, como é apresentado por Azevedo (2019)\nocite{azevedo2019analise} e Ramos (2020)\nocite{ramos2020analise}.

Além disso, o conceito de "white-label" tem ganhado popularidade no desenvolvimento de software, como explica Cann (2010). Cann define "white-labeling" como um produto ou serviço feito por uma compania e usado ou vendido por outras companias, esses produtos servem como base para um "rebrand" aonde é adicionada a identidade visual e marca da compania. Explica também que esse processo de "rebrand" pode ser realizado tanto pela empresa que produz quanto a que utiliza o "white-label".

Pensando nisso, esse trabalho busca aplicar o conceito de "white-label" para aplicações gamificadas no formato de quiz com o intuito de facilitar e/ou viabilizar o ensino de determinados temas ou assuntos para o público infantil.